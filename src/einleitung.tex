Im Rahmen des Moduls Computergrafik im Bachelor-Studiengang Software Engineering haben wir uns im vergangenen Semester mit einem faszinierenden Projekt auseinandergesetzt: der Entwicklung einer App für die HoloLens 2 unter Verwendung der Unity-Plattform.
Unser Ziel war es, eine Anwendung zu schreiben, die es den Benutzern ermöglicht, den Raum mittels virtueller Assets einzurichten und zu gestalten.
Diese Aufgabe eröffnete uns die spannende Welt der erweiterten Realität und bot uns die Möglichkeit, uns mit verschiedenen Konzepten und Technologien auseinanderzusetzen.


Leider mussten wir feststellen, dass wir unser Projekt nicht vollständig abschließen konnten.
Der Grund hierfür lag nicht etwa an mangelndem Engagement oder technischem Unvermögen, sondern an den Herausforderungen, die mit der Einrichtung der HoloLens 2 einhergingen.
Aufgrund der engen Zusammenarbeit mit den zuständigen Abteilungen für Informationstechnologie (IT) gestaltete sich die Konfiguration der HoloLens 2 zeitaufwendiger als erwartet.
Trotz dieser Schwierigkeiten haben wir unser Bestes gegeben, um die uns zur Verfügung stehende Zeit effektiv zu nutzen und uns intensiv mit den theoretischen Aspekten der Computergrafik und den Grundlagen der erweiterten Realität auseinanderzusetzen.


In dieser Ausarbeitung möchten wir die behandelten Themen vorstellen und unsere Erkenntnisse teilen.
Insbesondere legen wir den Fokus auf die Herausforderungen der Bewegungsfreiheit in der erweiterten Realität und die damit verbundenen Konzepte wie räumliches Denken (Spatial Cognition), sechs Freiheitsgrade (6DOF) und räumliche Anker (Spatial Anchors)
Dabei ist es uns wichtig zu betonen, dass wir uns nicht ausschließlich auf die HoloLens 2 beschränken, sondern auch einen allgemeinen Bezug zu den Konzepten herstellen möchten, die unabhängig von der spezifischen Hardware gelten.
Des Weiteren gehen wir auf die Modellierung von 3D-Objekten ein, die speziell für die HoloLens entwickelt werden müssen, um eine optimale Darstellung in der erweiterten Realität zu gewährleisten.


Wir hoffen, dass diese Ausarbeitung einen Einblick in die spannende Welt der Computergrafik im Kontext der erweiterten Realität bietet und für zukünftige Projekte auf diesem Gebiet von Nutzen sein wird.
Trotz der Herausforderungen, die wir während unseres Projekts erlebt haben, sind wir zuversichtlich, dass die erlangten Erkenntnisse einen wertvollen Beitrag zur Entwicklung innovativer Anwendungen in der erweiterten Realität leisten können.
