Im Rahmen des Moduls \enquote{Computergrafik} war es unser Ziel, ein Projekt umzusetzen, für das wir uns für den Titel \enquote{Raumgestalter} entschieden haben.

Unser Vorhaben bestand darin, die Hololens2 zu nutzen, um einen Raumgestalter zu entwickeln.
Unser Projekt ermöglichte es den Benutzern, durch ein immersives Erlebnis Objekte frei mit ihren eigenen Händen im Raum zu platzieren.
Dabei lag der Fokus nicht nur auf der beliebigen Platzierung, sondern auch auf der Möglichkeit, Gegenstände am Boden oder an Wänden zu fixieren.
Um dem Nutzer eine Vielfalt an Einrichtungsgegenständen zu bieten, wurden diese in verschiedenen Variationen zur Verfügung gestellt, die nahtlos ausgetauscht werden konnten.

Wir erkannten, dass die Verwendung der Hololens SDKs (Software Development Kits) von großem Vorteil war, da sie von Haus aus Funktionen wie Fingertracking (Gaze) und Umgebungstracking (Spatial Mapping) bereitstellten.
Allerdings stießen wir auch auf zwei wesentliche Herausforderungen:

\begin{enumerate}
    \item Die beschränkte Leistung der Hololens erschwerte das Rendering komplexer Modelle.
    \item Die Navigation und persistente Platzierung von Objekten in erweiterter Realität stellten eine herausfordernde Aufgabe dar.
\end{enumerate}

Bedauerlicherweise konnten wir unser Projekt nicht vollständig umsetzen und es kam nicht einmal richtig zum Start.
Grund hierfür war die langwierige Einrichtung der Hololens 2, für die wir auf die Unterstützung der IT-Abteilung in Gelsenkirchen angewiesen waren.
Die Integration der Hololens 2 in das WLAN-Netzwerk der Hochschule gestaltete sich kompliziert.

Da unsere praktischen Experimente stark eingeschränkt waren, liegt der Schwerpunkt dieser Ausarbeitung hauptsächlich auf theoretischen Betrachtungen.
Dennoch konnten wir dank der Microsoft-Dokumentation einige wichtige Details ermitteln und viele Schlussfolgerungen ziehen.

Im weiteren Verlauf dieser Arbeit wird erläutert, wie maximale Bewegungsfreiheit und eine optimale Grafik zu einem beeindruckenden, immersiven Augmented Reality-Erlebnis beitragen können.
