Die vorliegende Ausarbeitung behandelte die Entwicklung einer App für die HoloLens 2 mit Unity im Kontext des Moduls Computergrafik.
Obwohl wir aufgrund der Herausforderungen bei der Einrichtung der HoloLens 2 nicht in der Lage waren, unser Projekt vollständig abzuschließen, konnten wir dennoch wertvolle Erkenntnisse gewinnen und wichtige Themen im Bereich der erweiterten Realität (AR) und Computergrafik behandeln.

\paragraph{E}ine der Haupterkenntnisse aus unserem Projekt ist die Bedeutung der Bewegungsfreiheit in der erweiterten Realität.
    AR-Anwendungen eröffnen den Benutzern die Möglichkeit, sich frei in ihrem physischen Umfeld zu bewegen und virtuelle Objekte in Echtzeit zu interagieren.
    Die Berücksichtigung der sechs Freiheitsgrade (6~DOF) ist dabei von entscheidender Bedeutung, um ein immersives und realistisches AR-Erlebnis zu ermöglichen.
    Diese konzepte lassen sich alle stark von Konzepten aus der Biologie ableiten.


    Ein weiterer wichtiger Aspekt, den wir in unserer Ausarbeitung behandelt haben, ist die Verwendung von räumlichen Ankerpunkten (Spatial Anchors).
    Diese ermöglichen es, virtuelle Objekte an physische Standorte zu binden und somit eine konsistente Platzierung über mehrere Benutzersitzungen hinweg zu gewährleisten.
    Die korrekte Implementierung und Handhabung von räumlichen Ankerpunkten ist entscheidend, um die Benutzererfahrung zu verbessern und den Nutzen von AR-Anwendungen zu maximieren.

    Auch haben wir schlussfolgerungen und Ideen für die Render--Pipeline mit Spatial Anchors angeschaut.
    Dabei haben wir herausgefunden, dass bei einem wearable, durch die eingeschränkte größe und entsprechend eingeschränkte Grafikkarte, das Verteilen der GPU Last auf das Indoor Positioning System und die eigentliche Grafik aufzuteilen.
    Mit tricks, wie Zwischenframes, die die Kopfrotation ausgleichen, ist die geringe Framerate auch visuell ansprechend.

\paragraph{D}es Weiteren haben wir festgestellt, wie wichtig Grafikoptimierung für die Hololens ist. 
Da man oft nicht die Leistung oder Konfiguration der Endgeräte wählen kann, ist es umsowichtiger Lösungen innerhalb des Ramens dieser Limits zu finden.
Anhand von Limits kann man Innovationen finden die diese Limits umgehen oder auch um einiges strecken.
Durch diverse Grafikoptimierung kann man am ende größere Programme oder Szenen auf limitierter Hardware laufen lassen ohne größere Kompromisse.
Mithilfe von verringerung und das Zusammenfügen von Objekten und Texturen kann schon eine optimierung gewährleistet werden.
Auch durch das weglassen von überflüssigen Elementen hat man eine Optimisierungsmöglichkeit, die mehr Chancen für die kreativen Elemente der Entwickler in dem Programm lässt.

    %TODO

\paragraph*{}
    Abschließend können wir festhalten, dass die Welt der erweiterten Realität und Computergrafik ein enormes Potenzial birgt.
    Durch die richtige Berücksichtigung der Bewegungsfreiheit, 6~DOF, räumlicher Ankerpunkte und der korrekten Modellierung von 3D-Objekten können beeindruckende AR-Anwendungen entwickelt werden.
    Es ist jedoch wichtig, dass Entwicklerinnen und Entwickler sich kontinuierlich über neue Technologien und Konzepte informieren und mit den Herausforderungen und Chancen der AR-Entwicklung vertraut sind.
