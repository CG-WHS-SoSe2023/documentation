Die vorliegende Ausarbeitung behandelte die Entwicklung einer App für die HoloLens 2 mit Unity im Kontext des Moduls Computergrafik.
Obwohl wir aufgrund der Herausforderungen bei der Einrichtung der HoloLens 2 nicht in der Lage waren, unser Projekt vollständig abzuschließen, konnten wir dennoch wertvolle Erkenntnisse gewinnen und wichtige Themen im Bereich der erweiterten Realität (AR) und Computergrafik behandeln.


Eine der Haupterkenntnisse aus unserem Projekt ist die Bedeutung der Bewegungsfreiheit in der erweiterten Realität.
AR-Anwendungen eröffnen den Benutzern die Möglichkeit, sich frei in ihrem physischen Umfeld zu bewegen und virtuelle Objekte in Echtzeit zu interagieren.
Die Berücksichtigung der sechs Freiheitsgrade (6~DOF) ist dabei von entscheidender Bedeutung, um ein immersives und realistisches AR-Erlebnis zu ermöglichen.
Es ist wichtig, dass Entwicklerinnen und Entwickler diese Konzepte verstehen und in ihren Anwendungen entsprechend berücksichtigen.


Ein weiterer wichtiger Aspekt, den wir in unserer Ausarbeitung behandelt haben, ist die Verwendung von räumlichen Ankerpunkten (Spatial Anchors).
Diese ermöglichen es, virtuelle Objekte an physische Standorte zu binden und somit eine konsistente Platzierung über mehrere Benutzersitzungen hinweg zu gewährleisten.
Die korrekte Implementierung und Handhabung von räumlichen Ankerpunkten ist entscheidend, um die Benutzererfahrung zu verbessern und den Nutzen von AR-Anwendungen zu maximieren.


TODO: Fügen Sie eine kurze Zusammenfassung der behandelten Themen zur Modellierung von 3D-Objekten für die HoloLens hinzu.


Während unserer Recherche und Arbeit an diesem Projekt sind wir auf verschiedene Herausforderungen gestoßen.
Die Einrichtung der HoloLens 2 gestaltete sich zeitaufwendiger als erwartet und führte zu Verzögerungen bei der Entwicklung unserer App.
Es ist wichtig, dass Hochschulen und Organisationen, die AR-Projekte durchführen, ausreichende Ressourcen und Unterstützung für die technische Einrichtung solcher Geräte bereitstellen, um den Entwicklungsprozess zu optimieren.


Abschließend können wir festhalten, dass die Welt der erweiterten Realität und Computergrafik ein enormes Potenzial birgt.
Durch die richtige Berücksichtigung der Bewegungsfreiheit, 6~DOF, räumlicher Ankerpunkte und der korrekten Modellierung von 3D-Objekten können beeindruckende AR-Anwendungen entwickelt werden.
Es ist jedoch wichtig, dass Entwicklerinnen und Entwickler sich kontinuierlich über neue Technologien und Konzepte informieren und mit den Herausforderungen und Chancen der AR-Entwicklung vertraut sind.


TODO: Fügen Sie eine Zusammenfassung der Bedeutung des Projekts und einen Ausblick auf zukünftige Entwicklungen hinzu.


Insgesamt war dieses Projekt eine wertvolle Lernerfahrung für uns.
Wir konnten unser Wissen im Bereich der erweiterten Realität vertiefen und uns mit den Grundlagen der Computergrafik auseinandersetzen.
Trotz der Schwierigkeiten bei der Einrichtung der HoloLens 2 konnten wir wichtige Erkenntnisse gewinnen, die uns bei zukünftigen AR-Projekten von Nutzen sein werden.
Wir hoffen, dass unsere Ausarbeitung einen Beitrag zur weiteren Erforschung und Entwicklung von AR-Anwendungen leisten kann.
