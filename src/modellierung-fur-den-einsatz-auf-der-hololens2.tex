
\subsection{Limits der Hololens2}\label{subsec:Limits der Hololens2}

Auf der Hololens gibt es limits von objekten und deren anzahl an dreiecken.
Diese limits gibt es aufgrund der Prozessorleistung des eingebauten chips und der kühlungskapazität von dem eingebautem kühler.
Die Temperaturlimits sind nicht nur wegen dem schutz der Hardware sondern auch dem Tragekomfor bedingt. 

Microsoft schlägt für die Objekte in einer szene ein dreieck limit an. Diese limits sind für bis zu drei 
objekte in der szene bei bis zu 100.000\autocite{optimize_3d} dreiecke pro objekt. 
Diese limits sind nur empfehlungen die für den test wichtig sind da man an dieser Anzahl 
sehen kann wieweit man diese überschritten hat.
Durch tests kann abgeleitet werder in wiefern man sich an die vorgeschlagenen limits halten muss, bevor man 
an die Limits des Prozessors kommt. Somit kann man schon bei der erstellung einer szene schon einschätzen ob
die Leistung beeinflusst wird oder ob man noch im ramen liegt.

%¹microsoft seite mit limit werten
%Die von microsoft gegebenen limits¹ die für die anfänglichen test wichtig sind 
%lagen bei weniger als 100.000 dreiecke pro objekt 
%bei drei objekten in der szene. 

%Da das aber empfehlungen waren haben wir ein objekt mit vielen polygonen genommen,
%die diese anzahl deutlich überschreiten.
%durch diese tests kann man dann ableiten ob eine modelierte szene auch performant 
%auf dem endgerät laufen wird, oder unerwünschte
%performance veringerungen entstehen könnten. mit dieser information kann man 
%masnahmen einleiten die die performance so hoch wie 
%möglich hält.

\subsection{Testobjekte}\label{subsec:Testobjekte}


Für die Tests sind zwei objekte gewählt die jeweils über und unter dem vorgeschlagenem Limit \autocite{optimize_3d} sind.
Das kleinere Objekt ist ein affenkopf das standartmäsig mit dem Programm Blender mitgeliefert kommt.
Das Größere Objekt ist ein fraktal würfel der in Blender mit einem Plugin generiert wurde.
Beide objekte wurden jeweils als fbx datei exportiert.



%Als testobjekt wurde ein mengersponge fraktal und ein default affenkopf aus blender jeweils als fbx datei exportiert.
%die größe der datei des fraktals begab sich auf 9838KB und die des affenkopfs auf 57KB. die dreiekanzahl sind für das fraktal
%672.768 und für den affenkopf 968. diese objekte sind gewählt da sie auf zwei verschiedenen enden des limits sind.
%der affenkopf ist weniger als das limit und als test innerhalb der limitation und das fraktal als element weit oberhalb der limit grenze
%die auf der microsoft¹ seite zu finden ist.

\subsection{Test verfahren}\label{subsec:Test verfahren}

%erst wird getestet ob die objekte in dem vorinstallierten 3d viewer ladbar sind.
%nach diesen test werden die objekte in dem unity projekt importert und auf der hololens ausgeführt.
%die wichtigsten datenpunkte sind die fps, polygon anzahl und ob das objekt direkt angesegen wird oder nicht.
%nach diesen tests sind die limits für ein problemloses laufen des programms klar und man kann sehen in wie weit die limits, 
%die von microsoft vorgeschlagen sind eingegalten werden müssen.





\subsection{Test durchführung}\label{subsec:Test durchführung}


\subsection{Verbesserungsmethoden}\label{subsec:Verbesserungsmethoden}



%²microsoft seite vorschläge für optimisierung
%durch einfaches importieren des fraktals in das unity projekt konnte es angezeigt werden obwohl es sichtbare performance probleme
%generiert hat. 
%Durch externe optimisierungs optionen² kann die performance durchaus verbessert werden.
%man kann texturen kleiner² machen und mehrere² als eine textur importiern.
%Bei objekten die ein innenleben² haben oder objekte/flächen² die nicht gesehen werden, kann man diese einfach löschen so dass 
%die nicht erst in der laufzeit ignoriert werden müssen. eine weitere optimisierung ist dass man objekte die nicht voneinander
%getrennt werden einfach als ein objekt zusammengefügt² werden.

\subsection{Zukünftige verbesserungsmöglichkeiten}\label{subsec:Zukünftige verbesserungsmöglichkeiten}


%Es gibt einige möglichkeiten die performance für ar in der zukunft zu verbessern.
%durch ein desktop pc und kabelose verbinung kann man die performance verbessern und den komfort, da nur die anzeige seite auf
%dem headset unterstütst werden muss. falls keine möglichkeit für einen eigenen pc für streaming vorhanden ist kann auch ein online
%service mit der computing performance möglich sein. dadurch braucht das headset nur kleine leistung um das video streaming zu 
%unterstützen. durch verbesserungen in der mobilen chip technologien kann das headset in der zukunft kleiner, leichter und zur gleichen
%zeit performanter werden. dadurch würden die heutigen limitationen dann weiter zurückgeschoben werden.

